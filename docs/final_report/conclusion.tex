\chapter{Conclusion}

    Ce projet a été pour nous une source d'apprentissage inconsidérable. Ce dernier nous a permis de mieux comprendre une grande partie des matières enseignées au second semestre. De plus, il a été un support pour ces matières tout au long de cette période. Il a été parfois difficile de rester en phase avec les cours puisque nous avions besoin de plus de matériel pour continuer à travailler sur le projet.
    
    Bien qu'il fut compliqué de comprendre au départ les subtilités de ce projet, nous avons réussi en un temps raisonnable à nous approprier ce dernier et commencer à faire les premières fonctionnalités. 
    
    Nous avons, selon nous, fait un travail de qualité sur ce projet en nous concentrant sur un code propre, bien écrit, commenté, documenté (grâce à doxygen), et respectant un maximum de normes de bon sens. De plus, nous nous sommes efforcés de faire des commits utiles sur Git, pour une compréhension accrue de notre repository.
    
    Pour toutes ses raisons, nous pensons avoir réussi notre projet sur la forme en premier lieu puisque nous avons fait de notre mieux pour nous engager dans une démarche sérieuse et professionnelle. Bien sûr, nous avons aussi appris de ce projet sur cette démarche et savons ce que nous devrons refaire et ce que nous ne devrons pas
    
    En conclusion, nous avons tenté de mêler technique et gestion de projet, en utilisant un maximum de connaissances qui nous ont été enseignées durant nos différentes années d'études à l'EISTI, et sommes fiers du résultat obtenu. De plus, nous avons trouvé ce sujet très intéressant puisque techniquement élevé et aimerions retravailler sur un projet de ce type à l'avenir.