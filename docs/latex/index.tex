\#\-Readme

\subsection*{Documentation}

A special Doxygen documentation has been written for this project. You can read it as html by going to {\ttfamily \$\-P\-R\-O\-J\-E\-C\-T\-\_\-\-R\-O\-O\-T/docs/html/index.html}. You can also build your own documentation using the latex sources at {\ttfamily \$\-P\-R\-O\-J\-E\-C\-T\-\_\-\-R\-O\-O\-T/docs/latex}.

\subsection*{Prerequisite}

In order to compile and run this project you must have installed on your computer the followings \-: C\-Make, Boost, Open\-M\-P \begin{DoxyVerb}# apt-get install cmake libboost-all-dev libopenmpi1.6
\end{DoxyVerb}


\subsection*{Compile sources}

To compile sources, you will have to run the following lines \-: \begin{DoxyVerb}$ mkdir bin && cd bin && cmake .. && make
\end{DoxyVerb}


It will create a folder name 'bin', navigate to it, build the Makefile, and run the Makefile.

\subsection*{Populating}

Use -\/c or --client as argument on the scheduler's executable to start the client. It will populate the queue with your tasks. \begin{DoxyVerb}$ ./Scheduler -c
\end{DoxyVerb}


\subsection*{Scheduling}

Use -\/s or --sequential to start the scheduler in sequential mode. Use -\/p or --parallel to start in parallel. \begin{DoxyVerb}$ ./Scheduler -s
\end{DoxyVerb}


\subsubsection*{Algorithm}

\begin{DoxyVerb}TANT QUE non timeout FAIRE
    -- Une nouvelle tâche arrive
    SI aucun processeur ne peut contenir la tâche
        Attendre qu'un processeur se libère
    SINON
        SI un processeur est vide (pas de tâches en cours)
            lui attribuer automatiquement la tâche.
        SINON
            Récupérer le processeur le moins utilisé
            Lui assigner la tâche
        FINSI
    FINSI
FIN TQ\end{DoxyVerb}
 